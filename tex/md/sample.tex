\documentclass{article}
\usepackage{graphicx} % Required for inserting images

% Custom callout environment for Astro conversion
\newenvironment{callout}
{
    % Begin callout (could be styled in Astro)
}
{ 
    % End callout
}

\title{test file for astro conversion}
\author{wandke2}
\date{May 2025}

\begin{document}

\maketitle

\section{Section 1}
this should be a subsection

\begin{callout}
    This should be made into a callout card. We may add additional functionality later.
\end{callout}

\subsection{subsection 1.1}
this should be a subsubsection

\begin{callout}
    Inline math inside a callout: $E = mc^2$ and $a_i^2 + b_i^2 = c_i^2$
\end{callout}

\subsection{subsection 1.2}
this should be a subsubsection

\subsubsection{subsubsection 1.2.1}
this should be a subsubsubsection

\begin{callout}
    Display math inside a callout:
    \[
        \int_0^1 x^2 \, dx = \frac{1}{3}
    \]
    Inline next to display: $f(x) = x^2$
\end{callout}

\section{Section 2}

This should be an inline equation  $\alpha \sum_{n=0}^\infty a_n$  Immediately followed by text. 

\begin{callout}
    Test multiple inline equations: $x_1 + x_2 = x_3$, $F = ma$, $V = IR$
\end{callout}

This should be a displayed equation
\begin{equation}
    \alpha \sum_{n=0}^\infty a_n
\end{equation}
with text above and below

This should be a displayed equation
$$
    e^{i\pi} + 1 = 0
$$
with text above and below

This should be a displayed equation
\[
    f(x) = \sum_{n=0}^\infty \frac{f^{(n)}(a)}{n!}(x - a)^n
\]    
with text above and below

\begin{callout}
    Warning-style message: Make sure $\vec{E}$ is not confused with $E$.
    Use proper vector notation and units.
\end{callout}

\section{Section 3}

\begin{callout}
    Nested equations: Start with inline $p = mv$ and then:

    \[
        K = \frac{1}{2}mv^2
    \]

    Mix of chemistry too: $H_2O$, $CO_2$, $CH_3COOH$
\end{callout}

\begin{callout}
    Use of escaped \$ signs: You paid \$100 for a textbook. Not math!
\end{callout}

More inline math: $a^2 + b^2 = c^2$

And yet another equation:
\begin{equation}
    PV = nRT
\end{equation}

\end{document}