\section{Substances}
\subsection{\teal{Phases of matter}} %missing title
A phase of matter is homogeneous in both its molecular composition and the structure of its molecules with respect to one another. 

\paragraph{Pure Substance:} A pure substance is when a substance occurs in a single phase and is composed of a single type of molecule. 
\teal{\paragraph{Solid:} A substance that has a definite shape and volume is a solid.  The molecules in a solid are tightly packed, and cannot move past one another but vibrate.}
\teal{\paragraph{Liquid:} A substance that has a definite volume, but no definite shape is a liquid, taking on the shape of their container.  The molecules in a liquid are constantly moving yet still in contact with one another.}
\teal{\paragraph{Gas:} A substance with no definite shape or volume, gases take on the shape or volume of the container they are surrounded by.  Molecules can move freely and have negligible intermolecular interactions.}

\subsubsection{Phase Diagrams}
A phase diagram is like a map and predicts the phase of a substance according to its thermodynamic properties: $p$, $T$, $v$, $u$, and $h$. An example of a $pv$ liquid-vapor phase diagram for a pure substance is shown below:
\begin{figure}[h]
\begin{center}
\begin{tikzpicture}[auto,every text node part/.style={align=center},
	declare function={
        		curveL(\x) = -0.7*\x^2+10;
		curveR(\x) = -1*\x^2 + 0.3*x^2.5 + 10;
		TcurveL(\x) = 2*(\x+1.5)^4;
		TcurveR(\x) = -1*x+7.83+0.03*x^2;
  	}
	]
	
% left plot
	\node (ref) at (0,0) [draw=none, coordinate] {};
	\begin{axis} [at={(ref)},xmin=-5.5, ymin=0, xmax=8, ymax = 14, samples = 50, 
			ytick=\empty, yticklabels={}, ylabel style={rotate=-90},
			xtick=\empty, xticklabels={},
			ylabel={$p$}, xlabel={$v$},
			width = 8cm, height=7cm, axis y line = left, axis x line = bottom,
			thick, fill between/on layer={axis background}]
		%L+V dome
		\addplot[very thick,domain=-5:0, name path=L] {curveL(x)};	
		\path[name path=axL] (axis cs:-5, 0) -- (axis cs:0, 0);	
		\addplot[very thick,domain=0:7, name path=R] {curveR(x)};
		\path[name path=axR] (axis cs:0, 0) -- (axis cs:14, 0);
		\addplot [fill=Dteal]
		fill between [
			of = L and axL,
			soft clip={domain=-5:0},
		];
		\addplot [fill=Dteal]
		fill between [
			of = R and axR,
			soft clip={domain=0:14},
		];
				\addplot[color=Dteal] coordinates {(0,10) (0,0)}; % cover seam	
	
		%critical point
		\addplot[mark=*] coordinates {(0,10)};
		\addplot[dashed,thin,domain=-5.5:8,color=black!20] {10};
		
		
		% isotherm 
		\addplot[very thick,domain=-5:-2.74, Dlblue] {TcurveL(x)};	
		\addplot[very thick,domain=-2.74:3.4445,Dlblue] {4.74142};
		\addplot[very thick,domain=3.44449:14, Dlblue] {TcurveR(x)};	
		

		
		\node [below, color=black, yshift=-0.2cm] at (axis cs:0, 10) {critical \\[-5pt] point};
		\node [above, color=black] at (axis cs:6, 5) {vapor};
		\node [above, color=black] at (axis cs:-4.2, 5) {liquid};
		\node [color=Dlblue] at (axis cs:7, 3) {$T$};
		\node [above,color=white] at (axis cs:1, 1) {two-phase region};
		\node [above, color=black] at (axis cs:0, 12) {supercritical \\[-4pt]fluid};
		\draw [thick,stealth-] (axis cs:2.35863, 7) -- (axis cs:4, 9) node[right]{vapor\\[-4pt]  dome};
		%\node [above, color=black] at (axis cs:0.5, 0.2) {$\int_{V_1}^{V_2} p dV = $\hspace{6pt} \\ \hspace{18pt}$p (V_2-V_1)$};
		%\node [above, color=black] at (axis cs:0.5, 0.68) {process ${1\rightarrow2}$};

	\end{axis}
	\node (isolabel) at ($(ref)+(7.2,0.75)$) [draw=none,color=Dlblue]{example \\[-4pt]isotherm};

\end{tikzpicture}\end{center}
\caption{A $pv$ liquid-vapor phase diagram for a pure substance illustrating the critical point, vapor dome, and liquid and vapor regions. }\label{fig:PhaseDiag}
\end{figure}
\paragraph{Critical Point:} Point at which at lesser pressures the substance exists as either a liquid or vapor, and above exists as a supercritical fluid.
\paragraph{Vapor Dome:} Region within the dome-like curve known as the \textbf{two-phase region}, where some part of the substance is in a liquid phase, and some part is in a vapor phase. The substance is in a \textbf{liquid phase} when its $p$ and $v$ lie in the region to the left of the vapor dome and in a \textbf{vapor phase} when its $p$ and $v$ lie in the region to the right of the vapor dome.
\begin{itemize}
    \item If $v$ lies at the intersection of the isotherm and the left side of the vapor dome, the system is said to be in the \textbf{saturated liquid} state with specific volume $v_f$.
    \item If $v$ lies at the intersection of the isotherm and the right side of the vapor dome, the system is said to be in the \textbf{saturated vapor} state $v_g$.
    \item If the system is halfway between the liquid and vapor states (see Fig.~\ref{fig:PhaseDiag2phase}) such that the mass of the liquid portion $m_f$ equals the mass of the vapor portion $m_g$, then its \textbf{quality} $x$ is given by
        \begin{equation}
            x \equiv \frac{m_g}{m_f + m_g} = 0.5
        \end{equation}
\end{itemize}

\begin{figure}[h]
\begin{center}
\begin{tikzpicture}[auto,every text node part/.style={align=center},
	declare function={
        		curveL(\x) = -0.7*\x^2+10;
		curveR(\x) = -1*\x^2 + 0.3*x^2.5 + 10;
		TcurveL(\x) = 2*(\x+1.5)^4;
		TcurveR(\x) = -1*x+7.83+0.03*x^2;
  	}
	]
	
% left plot
	\node (ref) at (0,0) [draw=none, coordinate] {};
	\begin{axis} [at={(ref)},xmin=-5.5, ymin=0, xmax=8, ymax = 14, samples = 50, 
			ytick=\empty, yticklabels={}, ylabel style={rotate=-90},
			xtick=\empty, xticklabels={},
			ylabel={$p$}, xlabel={$v$},
			width = 8cm, height=7cm, axis y line = left, axis x line = bottom,
			thick, fill between/on layer={axis background}]
		%L+V dome
		\addplot[very thick,domain=-5:0, name path=L] {curveL(x)};	
		\path[name path=axL] (axis cs:-5, 0) -- (axis cs:0, 0);	
		\addplot[very thick,domain=0:7, name path=R] {curveR(x)};
		\path[name path=axR] (axis cs:0, 0) -- (axis cs:14, 0);
		\addplot [fill=Dteal]
		fill between [
			of = L and axL,
			soft clip={domain=-5:0},
		];
		\addplot [fill=Dteal]
		fill between [
			of = R and axR,
			soft clip={domain=0:14},
		];
		\addplot[color=Dteal] coordinates {(0,10) (0,0)}; % cover seam
		
		% points
		\addplot[mark=*] coordinates {(0,10)};
		\addplot[mark=*] coordinates {(-2.74,4.74142)} node[left]{$v_f$};
		\addplot[mark=*] coordinates {(3.4445,4.74142)} node[right]{$v_g$};
		\addplot[mark=*] coordinates {(0.352,4.74142)} node[above]{$x = 0.5$};
		
		% isotherm 
		\addplot[very thick,domain=-5:-2.74, Dlblue] {TcurveL(x)};	
		\addplot[ultra thick,domain=-2.74:3.4445,Dorange] {4.74142};
		\addplot[very thick,domain=3.44449:14, Dlblue] {TcurveR(x)};	
		
		% vf vg labels

		

		\node [above, color=black] at (axis cs:6, 5.5) {vapor};
		\node [above, color=black] at (axis cs:-4.2, 5.5) {liquid};
		\node [color=Dlblue] at (axis cs:7, 3) {$T$};
		\node [above,color=white] at (axis cs:1, 1) {two-phase region};
		\draw [thick,stealth-] (axis cs:2.35863, 7) -- (axis cs:4, 9) node[right]{vapor\\[-4pt]  dome};
		%\node [above, color=black] at (axis cs:0.5, 0.2) {$\int_{V_1}^{V_2} p dV = $\hspace{6pt} \\ \hspace{18pt}$p (V_2-V_1)$};
		%\node [above, color=black] at (axis cs:0.5, 0.68) {process ${1\rightarrow2}$};

	\end{axis}
	\node (isolabel) at ($(ref)+(7.2,0.75)$) [draw=none,color=Dlblue]{example \\[-4pt]isotherm};

\end{tikzpicture}\end{center}
\caption{A $pv$ liquid-vapor phase diagram for a pure substance illustrating thermodynamic properties that define the two-phase region. }\label{fig:PhaseDiag2phase}
\end{figure}
\paragraph{Supercritical Fluid:} Phase of the substance when existing above the critical point, and possesses characteristics similar to both liquids and gases, but strictly behaves as neither.
\paragraph{Isotherm:} For a given $p$ and $v$, the substance's $T$ lies on an isotherm. The segment of the isotherm within the vapor dome is \textbf{horizontal} because phase transitions occur at a constant $T$ and $p$.

\paragraph{State of a Substance:} If any two independent intensive variables are known ($T, v, p, x, h, u$), they fix the \textbf{state} of the substance. (Intensive variables are independent of mass. For example, specific volume $v$ is intensive. Volume $V=mv$ is not.)


\subsubsection{Critical Values and Ideal Gases}
Ideal gases are modeled as having internal energy entirely composed of kinetic energy due to the absence of any interactions between the particles of the gas, which occurs when:
\begin{enumerate}
    \item The particles are far away from each other, at lower pressure or density 
    \item the kinetic energy of the particles is so large as to render the energy due to interactions between them negligible, at high temperature
\end{enumerate}
\paragraph{Reduced Values:}
\emph{Low} and \emph{high} are relative terms and depend on the particular gas, specifically its phase behavior.  Reduced values are normalized with respect to their values at the critical point to describe where gases are far enough away from the vapor dome to behave ideally,
($p_c$, $v_c$, and $T_c$). The following expressions 
\begin{equation}
p_R = p/p_c \qquad T_R = T/T_c \qquad v_R' = \frac{v}{R(T_c/p_c)}
\end{equation}
define the \textbf{reduced pressure}, \textbf{reduced temperature}, and \textbf{pseudoreduced specific volume}.
\paragraph{Compressibility factor:} The compressibility factor, $Z$, for an ideal gas equals 1, meaning a gas behaves ideally at infinitely small $p$. Therefore, the more deviation of Z from unity the less ideally a gas behaves.
\begin{equation}\label{eq:Z}
Z = \frac{pv}{RT}
\end{equation}


\begin{tikzpicture}[auto,every text node part/.style={align=center}]
% left plot
	\node (ref) at (0,0) [draw=none, coordinate] {};
	\begin{axis} [at={(ref)},xmin=0, ymin=-5, xmax=10, ymax = 5, samples = 50, 
			ytick=0, yticklabels={$\bar{R}$}, ylabel style={rotate=-90},
			xtick=\empty, xticklabels={},
			ylabel={$\frac{pvM}{T}$}, xlabel={$p$},
			width = 5cm, height=5cm, axis y line = left, axis x line = bottom,
			thick, fill between/on layer={axis background},ylabel style = {at={(axis description cs:0,0.8)},anchor=south east},
			xlabel style = {at={(axis description cs:0.95,0)},anchor=north east}]

		\addplot[very thick,dashed] coordinates {(0,0) (3,1)};
		\addplot[very thick] coordinates {(3,1) (9,3)} node[above]{$T_1$};
		\addplot[very thick,dashed] coordinates {(0,0) (3,0.5)};
		\addplot[very thick] coordinates {(3,0.5) (9,1.5)} node[above]{$T_2$};
		\addplot[very thick,dashed] coordinates {(0,0) (3,-0.333)};
		\addplot[very thick] coordinates {(3,-0.333) (9,-1)} node[above]{$T_3$};
		\addplot[very thick,dashed] coordinates {(0,0) (3,-0.833)};
		\addplot[very thick] coordinates {(3,-0.833) (9,-2.5)} node[above]{$T_4$};

	\end{axis}

\end{tikzpicture}
%%%%%%%%%%%%%%%%%%%%%%%%%%%%%%%%%%%%%%%%%%%%%%%%%%%%%%%%%%%%%%%%%%%%%%%%%%%%%%%%%%%%%%%%%%%%%%%%%%%%%%%%%%%%%%%%%%%%%%%%%%%%%%%%%%%%%%%
\subsection{Ideal Gases}
A gas is a state of matter that has neither independent shape nor volume. When contained, the atoms or molecules that comprise it interact with every boundary of the container.
\subsubsection{Ideal Gas Assumptions}
An ideal gas is a hypothetical model of a gas that follows the assumptions that
\begin{itemize}
	\item molecules are infinitesimally small (just points in space)
    \begin{itemize}
        \item ideal gas molecules never collide with one another
        \item ideal gas molecules occupy no volume
    \end{itemize} 
	\item intermolecular forces are zero
    \begin{itemize}
        \item ideal gas molecules  have no potential energies between molecules
        \item internal energy is only a function of kinetic energy
    \end{itemize}
\end{itemize}
\vspace{12pt}

\begin{figure}[h]
\begin{center}
\begin{tikzpicture}[auto, >=latex']
	% Cylinder
  \node [cylinder,draw=black, thick,aspect=3,minimum height=8cm,minimum width=2.5cm,shape border rotate=0,cylinder uses custom fill, cylinder body     fill=Lteal,cylinder end fill=Dlblue] (A) {};
    % particle end
     \node [circle,draw=none, minimum width=5pt, ball color=Dteal, right=0.2cm of A.bottom] (p) {};
    \node[coordinate, right=0.5cm of p] (pend) {};
     	\draw [-latex, thick, draw=Dlblue] (p) -- (pend);
	    % particle end
     \node [circle,draw=none, minimum width=5pt, ball color=Dteal, right=1cm of A.bottom, yshift=1cm] (p2) {};
    \node[coordinate, right=0.5cm of p2] (pend2) {};
     	\draw [-latex, thick, draw=Dlblue] (p2) -- (pend2);
	    % particle end
     \node [circle,draw=none, minimum width=5pt, ball color=Dteal, right=3cm of A.bottom, yshift=-0.75cm] (p3) {};
    \node[coordinate, right=0.5cm of p3] (pend3) {};
     	\draw [-latex, thick, draw=Dlblue] (p3) -- (pend3);
	    % particle end
     \node [circle,draw=none, minimum width=5pt, ball color=Dteal, right=5cm of A.bottom, yshift=0.5cm] (p4) {};
    \node[coordinate, right=0.5cm of p4] (pend4) {};
     	\draw [-latex, thick, draw=Dlblue] (p4) -- (pend4);
	    % particle end
     \node [circle,draw=none, minimum width=5pt, ball color=Dteal, right=2cm of A.bottom, yshift=-0.2cm] (p5) {};
    \node[coordinate, left=0.5cm of p5] (pend5) {};
     	\draw [-latex, thick, draw=Dlblue] (p5) -- (pend5);
	    % particle end
     \node [circle,draw=none, minimum width=5pt, ball color=Dteal, right=4cm of A.bottom, yshift=0.9cm] (p6) {};
    \node[coordinate, left=0.5cm of p6] (pend6) {};
     	\draw [-latex, thick, draw=Dlblue] (p6) -- (pend6);
	    % particle end
     \node [circle,draw=none, minimum width=5pt, ball color=Dteal, right=6cm of A.bottom, yshift=-0.4cm] (p7) {};
    \node[coordinate, left=0.5cm of p7] (pend7) {};
     	\draw [-latex, thick, draw=Dlblue] (p7) -- (pend7);
     % other cylinder end
    \draw[solid, thick]
    let \p1 = ($ (A.after bottom) - (A.before bottom) $),
        \n1 = {0.5*veclen(\x1,\y1)-\pgflinewidth},
        \p2 = ($ (A.bottom) - (A.after bottom)!.5!(A.before bottom) $),
        \n2 = {veclen(\x2,\y2)-\pgflinewidth}
  in
    ([xshift=-\pgflinewidth] A.before bottom) arc [start angle=270, delta angle=180,
    x radius=\n2, y radius=\n1];	
    % Nodes for lines and labels
    \node[coordinate, above=0.1 cm of A.north] (xend) {};
    \node[coordinate, below=0.3 cm of A.before bottom] (D1) {};
    \node[coordinate, below=0.3 cm of A.after top] (D2) {};
    \node[coordinate, below=0.3 cm of A.south, xshift=-0.5cm] (D3) {};
    \node[coordinate, below=0.3 cm of A.south, xshift=0.5cm] (D4) {};
    \node[below=0 cm of A.south] {$|\vec{\textrm{v}}_x|t$};
    % Labels
	\node(area1) [draw=none, fill=none, left=0.1cm of A.top]{\textcolor{Lteal}{$A$}};
	\node(wall) [draw=none, fill=none, above=0.3cm of A.before top]{wall};
	\node(x)  [draw=none, fill=none, above=0.1cm of A.after bottom]{$x$};
	\node [draw=none, fill=none, above=1pt of pend] {$\vec{\textrm{v}}_x$};
	%\node(1) [draw=none, fill=none, right=0.1cm of B.east, yshift=0.7cm]{$\mathcal{B}_B$};
	% Lines and arrows
	\draw [->, thick] (wall) -- (A.before top);
	\draw [|->] (x.south) -- (xend);
	\draw [|-, thick] (D1) -- (D3);
	\draw [|-, thick] (D2) -- (D4);
\end{tikzpicture}
\end{center}
\caption{A sub-volume of gas next to a sub-section of the container wall having area, $A$.}\label{fig:IGequil}
\end{figure}

\subsubsection{Equation of State}
It has been shown empirically that an ideal gas obeys the \textbf{equation of state} 
\begin{equation}\label{eq:IGeos}
pV=n\overline{R}T
\end{equation}

\begin{equation*}
\overline{R} = \begin{cases} 
	&8.314 \textrm{ kJ/kmol}\cdot\textrm{K}\\
	&1.986 \textrm{ Btu/lbmol}\cdot\textrm{$^{\circ}$R}\\
	&1545 \textrm{ ft$\cdot$lbf/kmol}\cdot\textrm{$^{\circ}$R}\\
	\end{cases}
\end{equation*}
\subsubsection{Temperature Effects}
$\overline{R}$ defines the proportionality constant that converts the average translational kinetic energy of a single particle, $\langle KE\rangle $, to temperature ($N_A$ is Avagadro's number)
\begin{equation}
T = \frac{2}{3} \frac{N_A}{\overline{R}}\langle KE \rangle.
\end{equation}


Internal energy \underline{of an ideal gas} depends only on the total molecular kinetic energy.  Therefore, internal energy is solely a function of the temperature, $T$, and mass of the gas, $m$. The rate of change of the specific internal energy, $u$, where $u = U/m$, of an ideal gas with respect to $T$ is its constant volume specific heat, $c_v$
\begin{equation}
\frac{du}{dT} = c_v
\end{equation}
It follows that the change in internal energy of an ideal gas due to a change in temperature from $T_1$ to $T_2$ is given by
\begin{equation}
\Delta U = U_2 - U_1 = m\int_{T_1}^{T_2} c_v dT.
\end{equation}
%%%%%%%%%%%%%%%%%%%%%%%%%%%%%%%%%%%%%%%%%%%%%%%%%%%%%%%%%%%%%%%%%%%%%%%%%%%%%%%%%%%%%%%%%%%%%%%%%%%%%%%%%%%%%%%%%%%%%%%%%%%%%%%%%%%%
\subsection{\teal{Incompressible Fluids}}
In the \textbf{incompressible substance model}, the volume of a quantity of a substance cannot be compressed or expanded, thus its specific volume is constant. \subsubsection{Internal Energy of Incompressible Fluids}
Because the specific volume is not changing, the average intermolecular potential energy, which from the average distance between molecules, cannot change. Thus, the only contribution to an incompressible substance's internal energy comes from atomic and molecular kinetic energy, quantified via temperature. It follows that, like the ideal gas model, the \emph{internal energy of an incompressible substance is a function of temperature only} and 
    \begin{equation*}
        u = u(T)
    \end{equation*}
    \begin{equation*}
        \label{eqn:cvDefinition}
        \frac{du}{dT} = c_v.
    \end{equation*}
\subsubsection{Specific Heat of Incompressible Fluids}
For an incompressible substance, \emph{constant pressure specific heat} $c_p$ equals \emph{constant volume specific heat} $c_v$. $c_p$ is defined as the partial derivative of the enthalpy $h$, while pressure $p$ is held constant.
    \begin{equation*}
        c_p \equiv \left(\frac{\partial h}{\partial T}\right)_p.
    \end{equation*}
By definition $h = u+pv$, thus at constant $p$ and for an incompressible substance (constant $v$), we can write the partial derivative of $h$ with respect to $T$ as
    \begin{equation*}
        \left(\frac{\partial h}{\partial T}\right)_p = \left(\frac{\partial u}{\partial T}\right)_p = \frac{du}{dT} = c_v
    \end{equation*}
where the conversion from a partial to total derivative arises because $u = u(T)$ for an incompressible substance. Therefore $c_p = \left(\frac{\partial h}{\partial T}\right)_p = c_v$ for an incompressible substance.
    \begin{equation*}
        c_p = c_v = c
    \end{equation*}
\subsubsection{Specific Entropy Change of Incompressible Fluids}
For an \emph{ideal gas} with a given, \textbf{constant specific heat $c_v$}, its change in specific entropy from a state $1$ to a state $2$ is given by the expression
    \begin{equation*}
        s_2 - s_1 = c_v\ln\frac{T_2}{T_1}+ R \ln\frac{v_2}{v_1}
    \end{equation*}
where $R$ is the gas constant and $v$ is the specific volume. Assuming this ideal gas is incompressible, $v_2=v_1$.  Therefore, 
    \begin{equation*}
        s_2 - s_1 = c_v\ln\frac{T_2}{T_1}+ R \ln(1)
    \end{equation*}
the ratio of $\frac{v_2}{v_1}=1$ and the $ln(1)=0$, simplifying the change in specific entropy is 
    \begin{equation*}
        s_2 - s_1 = c\ln\frac{T_2}{T_1},
    \end{equation*}
as long as $c$ is constant.
    


