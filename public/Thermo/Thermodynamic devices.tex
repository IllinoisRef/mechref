\section{Thermodynamic Devices}
\subsection{Nozzle}
A flow passage that uses changes in cross-sectional area to increase the velocity (at the expense of pressure) of a fluid in the
direction of flow. (narrows)

\subsection{Diffuser}
A flow passage that uses changes in cross-sectional area to decrease the velocity of a fluid and increase its pressure in the
direction of flow. (narrows)

\subsection{Turbine}
A device for generating power as a result of the flow of fluid through it. Flow forces a set of blades to rotate a shaft via
which work is output.

\subsection{Compressor}
The reverse of a turbine. Work is input into the device in order to change the state of a gas in such a way as to make it flow (by an increase in its pressure).

\subsection{Pump}
Similar to a compressor. Work is input into the device in order to change the state of a liquid in such a way as to make it flow (by an increase in its pressure or elevation).

\subsection{Throttling device}
A device used to reduce the pressure of a flow. Enthalpy of the working fluid can be approximated as constant.(narrows)

\subsection{Heat Exchanger}
A device for transferring heat from one fluid substance to another.